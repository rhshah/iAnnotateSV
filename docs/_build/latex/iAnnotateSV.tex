% Generated by Sphinx.
\def\sphinxdocclass{report}
\documentclass[letterpaper,10pt,english]{sphinxmanual}
\usepackage[utf8]{inputenc}
\DeclareUnicodeCharacter{00A0}{\nobreakspace}
\usepackage{cmap}
\usepackage[T1]{fontenc}
\usepackage{babel}
\usepackage{times}
\usepackage[Bjarne]{fncychap}
\usepackage{longtable}
\usepackage{sphinx}
\usepackage{multirow}


\title{iAnnotateSV Documentation}
\date{March 04, 2015}
\release{0.0.1}
\author{Ronak Hasmukh Shah}
\newcommand{\sphinxlogo}{}
\renewcommand{\releasename}{Release}
\makeindex

\makeatletter
\def\PYG@reset{\let\PYG@it=\relax \let\PYG@bf=\relax%
    \let\PYG@ul=\relax \let\PYG@tc=\relax%
    \let\PYG@bc=\relax \let\PYG@ff=\relax}
\def\PYG@tok#1{\csname PYG@tok@#1\endcsname}
\def\PYG@toks#1+{\ifx\relax#1\empty\else%
    \PYG@tok{#1}\expandafter\PYG@toks\fi}
\def\PYG@do#1{\PYG@bc{\PYG@tc{\PYG@ul{%
    \PYG@it{\PYG@bf{\PYG@ff{#1}}}}}}}
\def\PYG#1#2{\PYG@reset\PYG@toks#1+\relax+\PYG@do{#2}}

\expandafter\def\csname PYG@tok@gd\endcsname{\def\PYG@tc##1{\textcolor[rgb]{0.63,0.00,0.00}{##1}}}
\expandafter\def\csname PYG@tok@gu\endcsname{\let\PYG@bf=\textbf\def\PYG@tc##1{\textcolor[rgb]{0.50,0.00,0.50}{##1}}}
\expandafter\def\csname PYG@tok@gt\endcsname{\def\PYG@tc##1{\textcolor[rgb]{0.00,0.27,0.87}{##1}}}
\expandafter\def\csname PYG@tok@gs\endcsname{\let\PYG@bf=\textbf}
\expandafter\def\csname PYG@tok@gr\endcsname{\def\PYG@tc##1{\textcolor[rgb]{1.00,0.00,0.00}{##1}}}
\expandafter\def\csname PYG@tok@cm\endcsname{\let\PYG@it=\textit\def\PYG@tc##1{\textcolor[rgb]{0.25,0.50,0.56}{##1}}}
\expandafter\def\csname PYG@tok@vg\endcsname{\def\PYG@tc##1{\textcolor[rgb]{0.73,0.38,0.84}{##1}}}
\expandafter\def\csname PYG@tok@m\endcsname{\def\PYG@tc##1{\textcolor[rgb]{0.13,0.50,0.31}{##1}}}
\expandafter\def\csname PYG@tok@mh\endcsname{\def\PYG@tc##1{\textcolor[rgb]{0.13,0.50,0.31}{##1}}}
\expandafter\def\csname PYG@tok@cs\endcsname{\def\PYG@tc##1{\textcolor[rgb]{0.25,0.50,0.56}{##1}}\def\PYG@bc##1{\setlength{\fboxsep}{0pt}\colorbox[rgb]{1.00,0.94,0.94}{\strut ##1}}}
\expandafter\def\csname PYG@tok@ge\endcsname{\let\PYG@it=\textit}
\expandafter\def\csname PYG@tok@vc\endcsname{\def\PYG@tc##1{\textcolor[rgb]{0.73,0.38,0.84}{##1}}}
\expandafter\def\csname PYG@tok@il\endcsname{\def\PYG@tc##1{\textcolor[rgb]{0.13,0.50,0.31}{##1}}}
\expandafter\def\csname PYG@tok@go\endcsname{\def\PYG@tc##1{\textcolor[rgb]{0.20,0.20,0.20}{##1}}}
\expandafter\def\csname PYG@tok@cp\endcsname{\def\PYG@tc##1{\textcolor[rgb]{0.00,0.44,0.13}{##1}}}
\expandafter\def\csname PYG@tok@gi\endcsname{\def\PYG@tc##1{\textcolor[rgb]{0.00,0.63,0.00}{##1}}}
\expandafter\def\csname PYG@tok@gh\endcsname{\let\PYG@bf=\textbf\def\PYG@tc##1{\textcolor[rgb]{0.00,0.00,0.50}{##1}}}
\expandafter\def\csname PYG@tok@ni\endcsname{\let\PYG@bf=\textbf\def\PYG@tc##1{\textcolor[rgb]{0.84,0.33,0.22}{##1}}}
\expandafter\def\csname PYG@tok@nl\endcsname{\let\PYG@bf=\textbf\def\PYG@tc##1{\textcolor[rgb]{0.00,0.13,0.44}{##1}}}
\expandafter\def\csname PYG@tok@nn\endcsname{\let\PYG@bf=\textbf\def\PYG@tc##1{\textcolor[rgb]{0.05,0.52,0.71}{##1}}}
\expandafter\def\csname PYG@tok@no\endcsname{\def\PYG@tc##1{\textcolor[rgb]{0.38,0.68,0.84}{##1}}}
\expandafter\def\csname PYG@tok@na\endcsname{\def\PYG@tc##1{\textcolor[rgb]{0.25,0.44,0.63}{##1}}}
\expandafter\def\csname PYG@tok@nb\endcsname{\def\PYG@tc##1{\textcolor[rgb]{0.00,0.44,0.13}{##1}}}
\expandafter\def\csname PYG@tok@nc\endcsname{\let\PYG@bf=\textbf\def\PYG@tc##1{\textcolor[rgb]{0.05,0.52,0.71}{##1}}}
\expandafter\def\csname PYG@tok@nd\endcsname{\let\PYG@bf=\textbf\def\PYG@tc##1{\textcolor[rgb]{0.33,0.33,0.33}{##1}}}
\expandafter\def\csname PYG@tok@ne\endcsname{\def\PYG@tc##1{\textcolor[rgb]{0.00,0.44,0.13}{##1}}}
\expandafter\def\csname PYG@tok@nf\endcsname{\def\PYG@tc##1{\textcolor[rgb]{0.02,0.16,0.49}{##1}}}
\expandafter\def\csname PYG@tok@si\endcsname{\let\PYG@it=\textit\def\PYG@tc##1{\textcolor[rgb]{0.44,0.63,0.82}{##1}}}
\expandafter\def\csname PYG@tok@s2\endcsname{\def\PYG@tc##1{\textcolor[rgb]{0.25,0.44,0.63}{##1}}}
\expandafter\def\csname PYG@tok@vi\endcsname{\def\PYG@tc##1{\textcolor[rgb]{0.73,0.38,0.84}{##1}}}
\expandafter\def\csname PYG@tok@nt\endcsname{\let\PYG@bf=\textbf\def\PYG@tc##1{\textcolor[rgb]{0.02,0.16,0.45}{##1}}}
\expandafter\def\csname PYG@tok@nv\endcsname{\def\PYG@tc##1{\textcolor[rgb]{0.73,0.38,0.84}{##1}}}
\expandafter\def\csname PYG@tok@s1\endcsname{\def\PYG@tc##1{\textcolor[rgb]{0.25,0.44,0.63}{##1}}}
\expandafter\def\csname PYG@tok@gp\endcsname{\let\PYG@bf=\textbf\def\PYG@tc##1{\textcolor[rgb]{0.78,0.36,0.04}{##1}}}
\expandafter\def\csname PYG@tok@sh\endcsname{\def\PYG@tc##1{\textcolor[rgb]{0.25,0.44,0.63}{##1}}}
\expandafter\def\csname PYG@tok@ow\endcsname{\let\PYG@bf=\textbf\def\PYG@tc##1{\textcolor[rgb]{0.00,0.44,0.13}{##1}}}
\expandafter\def\csname PYG@tok@sx\endcsname{\def\PYG@tc##1{\textcolor[rgb]{0.78,0.36,0.04}{##1}}}
\expandafter\def\csname PYG@tok@bp\endcsname{\def\PYG@tc##1{\textcolor[rgb]{0.00,0.44,0.13}{##1}}}
\expandafter\def\csname PYG@tok@c1\endcsname{\let\PYG@it=\textit\def\PYG@tc##1{\textcolor[rgb]{0.25,0.50,0.56}{##1}}}
\expandafter\def\csname PYG@tok@kc\endcsname{\let\PYG@bf=\textbf\def\PYG@tc##1{\textcolor[rgb]{0.00,0.44,0.13}{##1}}}
\expandafter\def\csname PYG@tok@c\endcsname{\let\PYG@it=\textit\def\PYG@tc##1{\textcolor[rgb]{0.25,0.50,0.56}{##1}}}
\expandafter\def\csname PYG@tok@mf\endcsname{\def\PYG@tc##1{\textcolor[rgb]{0.13,0.50,0.31}{##1}}}
\expandafter\def\csname PYG@tok@err\endcsname{\def\PYG@bc##1{\setlength{\fboxsep}{0pt}\fcolorbox[rgb]{1.00,0.00,0.00}{1,1,1}{\strut ##1}}}
\expandafter\def\csname PYG@tok@mb\endcsname{\def\PYG@tc##1{\textcolor[rgb]{0.13,0.50,0.31}{##1}}}
\expandafter\def\csname PYG@tok@ss\endcsname{\def\PYG@tc##1{\textcolor[rgb]{0.32,0.47,0.09}{##1}}}
\expandafter\def\csname PYG@tok@sr\endcsname{\def\PYG@tc##1{\textcolor[rgb]{0.14,0.33,0.53}{##1}}}
\expandafter\def\csname PYG@tok@mo\endcsname{\def\PYG@tc##1{\textcolor[rgb]{0.13,0.50,0.31}{##1}}}
\expandafter\def\csname PYG@tok@kd\endcsname{\let\PYG@bf=\textbf\def\PYG@tc##1{\textcolor[rgb]{0.00,0.44,0.13}{##1}}}
\expandafter\def\csname PYG@tok@mi\endcsname{\def\PYG@tc##1{\textcolor[rgb]{0.13,0.50,0.31}{##1}}}
\expandafter\def\csname PYG@tok@kn\endcsname{\let\PYG@bf=\textbf\def\PYG@tc##1{\textcolor[rgb]{0.00,0.44,0.13}{##1}}}
\expandafter\def\csname PYG@tok@o\endcsname{\def\PYG@tc##1{\textcolor[rgb]{0.40,0.40,0.40}{##1}}}
\expandafter\def\csname PYG@tok@kr\endcsname{\let\PYG@bf=\textbf\def\PYG@tc##1{\textcolor[rgb]{0.00,0.44,0.13}{##1}}}
\expandafter\def\csname PYG@tok@s\endcsname{\def\PYG@tc##1{\textcolor[rgb]{0.25,0.44,0.63}{##1}}}
\expandafter\def\csname PYG@tok@kp\endcsname{\def\PYG@tc##1{\textcolor[rgb]{0.00,0.44,0.13}{##1}}}
\expandafter\def\csname PYG@tok@w\endcsname{\def\PYG@tc##1{\textcolor[rgb]{0.73,0.73,0.73}{##1}}}
\expandafter\def\csname PYG@tok@kt\endcsname{\def\PYG@tc##1{\textcolor[rgb]{0.56,0.13,0.00}{##1}}}
\expandafter\def\csname PYG@tok@sc\endcsname{\def\PYG@tc##1{\textcolor[rgb]{0.25,0.44,0.63}{##1}}}
\expandafter\def\csname PYG@tok@sb\endcsname{\def\PYG@tc##1{\textcolor[rgb]{0.25,0.44,0.63}{##1}}}
\expandafter\def\csname PYG@tok@k\endcsname{\let\PYG@bf=\textbf\def\PYG@tc##1{\textcolor[rgb]{0.00,0.44,0.13}{##1}}}
\expandafter\def\csname PYG@tok@se\endcsname{\let\PYG@bf=\textbf\def\PYG@tc##1{\textcolor[rgb]{0.25,0.44,0.63}{##1}}}
\expandafter\def\csname PYG@tok@sd\endcsname{\let\PYG@it=\textit\def\PYG@tc##1{\textcolor[rgb]{0.25,0.44,0.63}{##1}}}

\def\PYGZbs{\char`\\}
\def\PYGZus{\char`\_}
\def\PYGZob{\char`\{}
\def\PYGZcb{\char`\}}
\def\PYGZca{\char`\^}
\def\PYGZam{\char`\&}
\def\PYGZlt{\char`\<}
\def\PYGZgt{\char`\>}
\def\PYGZsh{\char`\#}
\def\PYGZpc{\char`\%}
\def\PYGZdl{\char`\$}
\def\PYGZhy{\char`\-}
\def\PYGZsq{\char`\'}
\def\PYGZdq{\char`\"}
\def\PYGZti{\char`\~}
% for compatibility with earlier versions
\def\PYGZat{@}
\def\PYGZlb{[}
\def\PYGZrb{]}
\makeatother

\renewcommand\PYGZsq{\textquotesingle}

\begin{document}

\maketitle
\tableofcontents
\phantomsection\label{index::doc}

\begin{quote}\begin{description}
\item[{Author}] \leavevmode
Ronak H Shah

\item[{Contact}] \leavevmode
\href{mailto:rons.shah@gmail.com}{rons.shah@gmail.com}

\item[{Source code}] \leavevmode
\href{http://github.com/rhshah/iAnnotateSV}{http://github.com/rhshah/iAnnotateSV}

\item[{License}] \leavevmode
\href{http://www.apache.org/licenses/LICENSE-2.0}{Apache License 2.0}

\end{description}\end{quote}

iAnnotateSV is a Python library and command-line software toolkit to annotate and
visualize structural variants detected from Next Generation DNA sequencing data. This works for majority is just a re-writing of a tool called dRanger\_annotate written in matlab by Mike Lawrence at Broad Institue.
But it also has some additional functionality and control over the annotation w.r.t the what transcripts to be used for annotation.
It is designed for use with hybrid capture, including both whole-exome and custom target panels, and
short-read sequencing platforms such as Illumina.

Contents:


\chapter{iAnnotateSV}
\label{modules::doc}\label{modules:iannotatesv-annotation-of-structural-variants-detected-from-ngs}\label{modules:iannotatesv}

\section{iAnnotateSV package}
\label{iAnnotateSV:iannotatesv-package}\label{iAnnotateSV::doc}

\subsection{Module \texttt{iAnnotateSV} contents}
\label{iAnnotateSV:module-iannotatesv-contents}

\subsection{Submodules}
\label{iAnnotateSV:submodules}

\subsection{\texttt{AnnotateEachBreakpoint} module}
\label{iAnnotateSV:annotateeachbreakpoint-module}\begin{itemize}
\item {} \begin{description}
\item[{This module will annotate each breakpoint taking in:}] \leavevmode\begin{itemize}
\item {} 
\textbf{chr} : chromosome for the event,

\item {} 
\textbf{pos} : position in the chromosome for the event,

\item {} 
\textbf{str} : direction of the reads for the event{[}either 0 or 1{]},

\item {} 
\textbf{referenceDataframe} : a pandas data-frame that will store reference information

\end{itemize}
\begin{quote}\begin{description}
\item[{Example}] \leavevmode
\code{AnnotateEachBreakpoint(chr1,pos1,str1,refDF)}

\end{description}\end{quote}

\end{description}

\end{itemize}


\subsection{\texttt{PredictFunction} module}
\label{iAnnotateSV:predictfunction-module}\begin{itemize}
\item {} 
This module will predict the function of each annotated breakpoint

\item {} \begin{description}
\item[{It takes two pandas series which has following information:}] \leavevmode\begin{itemize}
\item {} 
\textbf{gene} : Gene for the event,

\item {} 
\textbf{transcript} : Transcript used for the event,

\item {} 
\textbf{site} : Explanation for site where the event occurs,

\item {} 
\textbf{zone} : Where does the event occur {[} 1=exon, 2=intron, 3=3'-UTR, 4=5'-UTR, 5=promoter {]},

\item {} 
\textbf{strand} : Direction of the transcript,

\item {} 
\textbf{str} : Direction of the read,

\item {} 
\textbf{intronnum} : Which intron the event occurs if the event is in intron,

\item {} 
\textbf{intronframe} : What is the frame of the intron where the event is occuring.

\end{itemize}
\begin{quote}\begin{description}
\item[{Example}] \leavevmode
\code{ann1S = pandas.Series({[}gene1,transcript1,site1,zone1,strand1,str1,intronnum1,intronframe1{]},index={[}'gene1', 'transcript1', 'site1', 'zone1', 'txstrand1', 'readstrand1', 'intronnum1','intronframe1'{]})}

\code{ann2S = pandas.Series({[}gene2,transcript2,site2,zone2,strand2,str2,intronnum2,intronframe2{]},index={[}'gene2', 'transcript2', 'site2', 'zone2', 'txstrand2', 'readstrand2', 'intronnum2','intronframe2'{]})}

So \textbf{ann1S} \& \textbf{ann2S} are series that will go to PredictFuntionForSV()

\code{PredictFunctionForSV(ann1S,ann2S)}

\end{description}\end{quote}

\end{description}

\end{itemize}


\subsection{\texttt{helper} module}
\label{iAnnotateSV:helper-module}\begin{itemize}
\item {} \begin{description}
\item[{This module has multiple submodules}] \leavevmode\begin{enumerate}
\item {} 
\textbf{ReadFile()}
\begin{itemize}
\item {} 
This will read a tab-delimited file into a panadas dataframe

\end{itemize}

\item {} 
\textbf{ExtendPromoterRegion()}
\begin{itemize}
\item {} 
This will extend the promoter region to a given length

\end{itemize}

\item {} 
\textbf{bp2str()}
\begin{itemize}
\item {} 
This will convert base pair information to string information

\end{itemize}

\end{enumerate}

\end{description}

\end{itemize}


\subsection{\texttt{iAnnotateSV} module}
\label{iAnnotateSV:iannotatesv-module}\begin{itemize}
\item {} 
This module is the driver module, it takes user information and runs all other module to produce proper structural variant annotation

\end{itemize}

\textbf{usage: iAnnotateSV.py {[}options{]}}

\textbf{Annotate SV based on a specific human reference}
\begin{description}
\item[{\textbf{optional arguments:}}] \leavevmode\begin{optionlist}{3cm}
\item [-h, -{-}help]  
show this help message and exit
\item [-v, -{-}verbose]  
make lots of noise {[}default{]}
\item [-r hg19, -{-}refFileVersion hg19]  
Which human reference file to be used, hg18,hg19 or
hg38
\end{optionlist}
\begin{description}
\item[{-o outfile, --outputFile out.txt}] \leavevmode
Full path with for the output file

\item[{-i inputSVfile, --svFile svfile.txt}] \leavevmode
Location of the structural variants file to annotate

\item[{-d distance, --distance 3000}] \leavevmode
Distance used to extend the promoter region

\end{description}
\begin{optionlist}{3cm}
\item [-a, -{-}autoSelect]  
Auto Select which transcript to be used{[}default{]}
\end{optionlist}
\begin{description}
\item[{-c canonicalExonsFile, --canonicalTranscripts canonicalExons.txt}] \leavevmode
Location of canonical transcript list for each gene.
Use only if you want the output for specific
transcripts for each gene.

\end{description}

\end{description}


\chapter{Citation}
\label{index:citation}
We are in the process of publishing a manuscript describing iAnnotateSV as part of the Structural Variant Detection framework.
If you use this software in a publication, for now, please cite our website \href{http://github.com/rhshah/iAnnotateSV}{iAnnotateSV}.


\chapter{Indices and tables}
\label{index:indices-and-tables}\begin{itemize}
\item {} 
\emph{genindex}

\item {} 
\emph{modindex}

\item {} 
\emph{search}

\end{itemize}



\renewcommand{\indexname}{Index}
\printindex
\end{document}
