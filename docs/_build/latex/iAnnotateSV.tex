%% Generated by Sphinx.
\def\sphinxdocclass{report}
\documentclass[letterpaper,10pt,english]{sphinxmanual}
\ifdefined\pdfpxdimen
   \let\sphinxpxdimen\pdfpxdimen\else\newdimen\sphinxpxdimen
\fi \sphinxpxdimen=.75bp\relax

\usepackage[utf8]{inputenc}
\ifdefined\DeclareUnicodeCharacter
 \ifdefined\DeclareUnicodeCharacterAsOptional
  \DeclareUnicodeCharacter{"00A0}{\nobreakspace}
  \DeclareUnicodeCharacter{"2500}{\sphinxunichar{2500}}
  \DeclareUnicodeCharacter{"2502}{\sphinxunichar{2502}}
  \DeclareUnicodeCharacter{"2514}{\sphinxunichar{2514}}
  \DeclareUnicodeCharacter{"251C}{\sphinxunichar{251C}}
  \DeclareUnicodeCharacter{"2572}{\textbackslash}
 \else
  \DeclareUnicodeCharacter{00A0}{\nobreakspace}
  \DeclareUnicodeCharacter{2500}{\sphinxunichar{2500}}
  \DeclareUnicodeCharacter{2502}{\sphinxunichar{2502}}
  \DeclareUnicodeCharacter{2514}{\sphinxunichar{2514}}
  \DeclareUnicodeCharacter{251C}{\sphinxunichar{251C}}
  \DeclareUnicodeCharacter{2572}{\textbackslash}
 \fi
\fi
\usepackage{cmap}
\usepackage[T1]{fontenc}
\usepackage{amsmath,amssymb,amstext}
\usepackage{babel}
\usepackage{times}
\usepackage[Bjarne]{fncychap}
\usepackage[dontkeepoldnames]{sphinx}

\usepackage{geometry}

% Include hyperref last.
\usepackage{hyperref}
% Fix anchor placement for figures with captions.
\usepackage{hypcap}% it must be loaded after hyperref.
% Set up styles of URL: it should be placed after hyperref.
\urlstyle{same}

\addto\captionsenglish{\renewcommand{\figurename}{Fig.}}
\addto\captionsenglish{\renewcommand{\tablename}{Table}}
\addto\captionsenglish{\renewcommand{\literalblockname}{Listing}}

\addto\captionsenglish{\renewcommand{\literalblockcontinuedname}{continued from previous page}}
\addto\captionsenglish{\renewcommand{\literalblockcontinuesname}{continues on next page}}

\addto\extrasenglish{\def\pageautorefname{page}}

\setcounter{tocdepth}{1}



\title{iAnnotateSV Documentation}
\date{Jan 09, 2018}
\release{0.0.1}
\author{Ronak Hasmukh Shah}
\newcommand{\sphinxlogo}{\vbox{}}
\renewcommand{\releasename}{Release}
\makeindex

\begin{document}

\maketitle
\sphinxtableofcontents
\phantomsection\label{\detokenize{index::doc}}

\begin{quote}\begin{description}
\item[{Author}] \leavevmode
Ronak H Shah

\item[{Contact}] \leavevmode
\sphinxhref{mailto:rons.shah@gmail.com}{rons.shah@gmail.com}

\item[{Source code}] \leavevmode
\sphinxurl{http://github.com/rhshah/iAnnotateSV}

\item[{License}] \leavevmode
\sphinxhref{http://www.apache.org/licenses/LICENSE-2.0}{Apache License 2.0}

\end{description}\end{quote}

iAnnotateSV is a Python library and command-line software toolkit to annotate and
visualize structural variants detected from Next Generation DNA sequencing data. This works for majority is just a re-writing of a tool called dRanger\_annotate written in matlab by Mike Lawrence at Broad Institue.
But it also has some additional functionality and control over the annotation w.r.t the what transcripts to be used for annotation.
It is designed for use with hybrid capture, including both whole-exome and custom target panels, and
short-read sequencing platforms such as Illumina.
\sphinxhref{https://landscape.io/github/rhshah/iAnnotateSV/master}{\sphinxincludegraphics{{/Users/rshah22/Documents/GitHub/iAnnotateSV/docs/_build/doctrees/images/https/landscape.io/github/rhshah/iAnnotateSV/master/landscape.svg/style=flat/landscape}.svg}}\sphinxhref{https://zenodo.org/badge/latestdoi/18929/rhshah/iAnnotateSV}{\sphinxincludegraphics{{/Users/rshah22/Documents/GitHub/iAnnotateSV/docs/_build/doctrees/images/https/zenodo.org/badge/18929/rhshah/iAnnotateSV.svg/iAnnotateSV}.svg}}\sphinxhref{https://travis-ci.org/rhshah/iAnnotateSV}{\sphinxincludegraphics{{/Users/rshah22/Documents/GitHub/iAnnotateSV/docs/_build/doctrees/images/https/travis-ci.org/rhshah/iAnnotateSV.svg/branch=master/iAnnotateSV}.svg}}\sphinxhref{https://codecov.io/gh/rhshah/iAnnotateSV}{\sphinxincludegraphics{{/Users/rshah22/Documents/GitHub/iAnnotateSV/docs/_build/doctrees/images/https/codecov.io/gh/rhshah/iAnnotateSV/branch/master/graph/badge.svg/badge}.svg}}
Contents:


\chapter{iAnnotateSV}
\label{\detokenize{modules::doc}}\label{\detokenize{modules:iannotatesv-annotation-of-structural-variants-detected-from-ngs}}\label{\detokenize{modules:iannotatesv}}

\section{iAnnotateSV package}
\label{\detokenize{iAnnotateSV:iannotatesv-package}}\label{\detokenize{iAnnotateSV::doc}}

\section{Required Packages}
\label{\detokenize{iAnnotateSV:required-packages}}
We require that you install:
\begin{quote}\begin{description}
\item[{pandas}] \leavevmode
\sphinxhref{http://pandas.pydata.org/}{v0.16.2}

\item[{biopython}] \leavevmode
\sphinxhref{http://biopython.org/wiki/Main\_Page}{v1.65}

\item[{Pillow}] \leavevmode
\sphinxhref{https://pypi.python.org/pypi/Pillow/3.4.2}{v3.4.2}

\item[{reportlab}] \leavevmode
\sphinxhref{https://pypi.python.org/pypi/reportlab/3.3.0}{v3.3.0}

\item[{coloredlogs}] \leavevmode
\sphinxhref{https://pypi.python.org/pypi/coloredlogs}{v5.2}

\end{description}\end{quote}


\section{Quick Usage}
\label{\detokenize{iAnnotateSV:quick-usage}}
If you know python I have created a small test script in /iAnnotateSV/test directory it runs a test on existing code and compares the result with the output file.
\begin{description}
\item[{Else To Run:}] \leavevmode\begin{itemize}
\item {} 
If you want to run with default options:

\end{itemize}

\sphinxcode{python iAnnotateSV.py -i svFile.txt -ofp outputfilePrefix -o /path/to/output/dir -r hg19 -d 3000}
\begin{itemize}
\item {} 
If you want to run with your own transcripts:

\end{itemize}

\sphinxcode{python iAnnotateSV.py -i svFile.txt -ofp outputfilePrefix -o /path/to/output/dir -r hg19 -d 3000 -c canonicalTranscripts.txt}
\begin{itemize}
\item {} 
If you want to run with your own transcripts \& make plots:

\end{itemize}

\sphinxcode{python iAnnotateSV.py -i svFile.txt -ofp outputfilePrefix -o /path/to/output/dir -r hg19 -d 3000 -c canonicalTranscripts.txt -u uniprot.txt -p}

\end{description}

\fvset{hllines={, ,}}%
\begin{sphinxVerbatim}[commandchars=\\\{\}]
usage: iAnnotateSV.py \PYG{o}{[}options\PYG{o}{]}

**Annotate SV based on a specific human reference**

**optional arguments:**
    \PYGZhy{}h, \PYGZhy{}\PYGZhy{}help            show this \PYG{n+nb}{help} message and \PYG{n+nb}{exit}
    \PYGZhy{}v, \PYGZhy{}\PYGZhy{}verbose         make lots of noise \PYG{o}{[}default\PYG{o}{]}
    \PYGZhy{}r hg19, \PYGZhy{}\PYGZhy{}refFileVersion hg19
                        Which human reference file to be used, hg18,hg19 or
                        hg38
    \PYGZhy{}ofp test, \PYGZhy{}\PYGZhy{}outputFilePrefix \PYG{n+nb}{test}
                        Prefix \PYG{k}{for} the output file
    \PYGZhy{}o /somedir, \PYGZhy{}\PYGZhy{}outputDir /somedir
                        Full Path to the output dir
    \PYGZhy{}i svfile.txt, \PYGZhy{}\PYGZhy{}svFile svfile.txt
                        Location of the structural variants file to annotate
    \PYGZhy{}d \PYG{l+m}{3000}, \PYGZhy{}\PYGZhy{}distance \PYG{l+m}{3000}
                        Distance used to extend the promoter region
    \PYGZhy{}a, \PYGZhy{}\PYGZhy{}autoSelect      Auto Select which transcript to be used\PYG{o}{[}default\PYG{o}{]}
    \PYGZhy{}c canonicalExons.txt, \PYGZhy{}\PYGZhy{}canonicalTranscripts canonicalExons.txt
                        Location of canonical transcript list \PYG{k}{for} each gene.
                        Use only \PYG{k}{if} you want the output \PYG{k}{for} specific
                        transcripts \PYG{k}{for} each gene.
    \PYGZhy{}p, \PYGZhy{}\PYGZhy{}plotSV          Plot the structural variant in question\PYG{o}{[}default\PYG{o}{]}
    \PYGZhy{}u uniprot.txt, \PYGZhy{}\PYGZhy{}uniprotFile uniprot.txt
                        Location of UniProt list contain information \PYG{k}{for}
                        protein domains. Use only \PYG{k}{if} you want to plot the
                        structural variant
    \PYGZhy{}rr RepeatRegionFile.tsv, \PYGZhy{}\PYGZhy{}repeatFile RepeatRegionFile.tsv
                        Location of the Repeat Region Bed File
    \PYGZhy{}dgv DGvFile.tsv, \PYGZhy{}\PYGZhy{}dgvFile DGvFile.tsv
                        Location of the Database of Genomic Variants Bed File
    \PYGZhy{}cc CosmicConsensus.tsv, \PYGZhy{}\PYGZhy{}cosmicConsensusFile CosmicConsensus.tsv
                        Location of the Cosmic Consensus TSV file
    \PYGZhy{}cct CosmicFusionCounts.tsv, \PYGZhy{}\PYGZhy{}cosmicCountsFile CosmicConsensus.tsv
                        Location of the Cosmic Fusion Counts TSV file
\end{sphinxVerbatim}

Input file format is a tab-delimited file containing:

chr1  pos1  str1  chr2  pos2  str2

as the header and where:
\begin{itemize}
\item {} 
\sphinxstylestrong{chr1:} Its the chromosome name for first break point {[}1,2,3,4,5,6,7 etc..{]},

\item {} 
\sphinxstylestrong{pos1:} Its the chromosome loaction for first break point {[}1-based{]},

\item {} 
\sphinxstylestrong{str1:} Its the read direction for the first break point {[}0=top/plus/reference, 1=bottom/minus/complement{]},

\item {} 
\sphinxstylestrong{chr2:} Its the chromosome name for second break point {[}1,2,3,4,5,6,7 etc..{]},

\item {} 
\sphinxstylestrong{pos2:} Its the chromosome loaction for second break point {[}1-based{]},

\item {} 
\sphinxstylestrong{str2:} Its the read direction for the second break point {[}0=top/plus/reference, 1=bottom/minus/complement{]},

\end{itemize}

Output file will is a tab-delimited file containing:

chr1  pos1  str1  chr2  pos2  str2  gene1 transcript1 site1 gene2 transcript2 site2 fusion

as the header and where:
\begin{itemize}
\item {} 
\sphinxstylestrong{chr1} : Its the chromosome name for first break point {[}1,2,3,4,5,6,7 etc..{]},

\item {} 
\sphinxstylestrong{pos1} : Its the chromosome loaction for first break point {[}1-based{]},

\item {} 
\sphinxstylestrong{str1} : Its the read direction for the first break point {[}0=top/plus/reference, 1=bottom/minus/complement{]},

\item {} 
\sphinxstylestrong{chr2} : Its the chromosome name for second break point {[}1,2,3,4,5,6,7 etc..{]},

\item {} 
\sphinxstylestrong{pos2} : Its the chromosome loaction for second break point {[}1-based{]},

\item {} 
\sphinxstylestrong{str2} : Its the read direction for the second break point {[}0=top/plus/reference, 1=bottom/minus/complement{]},

\item {} 
\sphinxstylestrong{gene1} : Gene for the first break point,

\item {} 
\sphinxstylestrong{transcript1} : Transcript used for the first breakpoint,

\item {} 
\sphinxstylestrong{site1} : Explanation of the site where the first breakpoint occurs {[}Example=\textgreater{}Intron of EWSR1(+):126bp after exon 10{]},

\item {} 
\sphinxstylestrong{kinasedomain1} : Explanation of the site where the first breapoint involves a Kinase Domain or not{[}Example=\textgreater{}Partial Kinase Domain Included{]}

\item {} 
\sphinxstylestrong{gene2} : Gene for the second break point,

\item {} 
\sphinxstylestrong{transcript2} : Transcript used for the second breakpoint,

\item {} 
\sphinxstylestrong{site2} : Explanation of the site where the second breakpoint occurs {[}Example=\textgreater{}Intron of ERG(-):393bp after exon 4{]},

\item {} 
\sphinxstylestrong{kinasedomain2} : Explanation of the site where the second breapoint involves a Kinase Domain or not{[}Example=\textgreater{}Partial Kinase Domain Included{]}

\item {} 
\sphinxstylestrong{fusion} : Explanation if the evnet leads to fusion or not. {[}Example=\textgreater{}Protein Fusion: in frame  \{EWSR1:ERG\}{]}

\item {} 
\sphinxstylestrong{Cosmic\_Fusion\_Counts} : Number of Counts for the Events from Cosmic Fusion Results

\item {} 
\sphinxstylestrong{repName-repClass-repFamily:-site1} : Repeat Region Annotation for site 1

\item {} 
\sphinxstylestrong{repName-repClass-repFamily:-site2} : Repeat Region Annotation for site 2

\item {} 
\sphinxstylestrong{CC\_Chr\_Band} : Cosmic Cancer Census Chromosome Band

\item {} 
\sphinxstylestrong{CC\_Tumour\_Types(Somatic)} : Cosmic Cancer Census Tumor Type in Somatic Samples

\item {} 
\sphinxstylestrong{CC\_Cancer\_Syndrome} : Cosmic Cancer Census Cancer Syndrome the genes are related to.

\item {} 
\sphinxstylestrong{CC\_Mutation\_Type} : Cosmic Cancer Census Mutation Types the Genes are related to.

\item {} 
\sphinxstylestrong{CC\_Translocation\_Partner} : Cosmic Cancer Census Translocation Partners for the gene.

\item {} 
\sphinxstylestrong{DGv\_Name-DGv\_VarType-site1} : Database of Genomic Variants annotation for site 1

\item {} 
\sphinxstylestrong{DGv\_Name-DGv\_VarType-site} : Database of Genomic Variants annotation for site 2

\end{itemize}
\begin{quote}\begin{description}
\item[{Example Plot}] \leavevmode
\noindent{\hspace*{\fill}\scalebox{1.000000}{\sphinxincludegraphics[width=300\sphinxpxdimen,height=300\sphinxpxdimen]{{EWSR1-chr22_29688289_ERG-chr21_39775034_Translocation}.jpg}}\hspace*{\fill}}

\end{description}\end{quote}

The above plot shows the following:
\begin{itemize}
\item {} 
There are three tracks for each break point. The first three tracks belong to breakpoint 1 and second three tracks belong to breakpoint 2.

\item {} \begin{description}
\item[{Thre three tracks are:}] \leavevmode\begin{itemize}
\item {} \begin{description}
\item[{Gene Model Track:}] \leavevmode\begin{itemize}
\item {} 
Displays \sphinxstylestrong{Exons} as \sphinxstylestrong{brown} and there direction with exons in arrow formation.

\item {} 
In \sphinxstylestrong{orange} it displays the \sphinxstylestrong{breakpoint description}.

\end{itemize}

\end{description}

\item {} \begin{description}
\item[{Alignment Track:}] \leavevmode\begin{itemize}
\item {} 
Displays the direction of the reads for the breakpoint. Also displays the \sphinxstylestrong{co-ordinate} in \sphinxstylestrong{Purple.}

\item {} 
Read in \sphinxstylestrong{Positive} direction are \sphinxstylestrong{Blue} and \sphinxstylestrong{Negative} direction are \sphinxstylestrong{Red}

\end{itemize}

\end{description}

\item {} \begin{description}
\item[{Protein Domain Track:}] \leavevmode\begin{itemize}
\item {} 
Displays the \sphinxstylestrong{Protein Domain} as \sphinxstylestrong{green boxes} with there names in green.

\end{itemize}

\end{description}

\end{itemize}

\end{description}

\end{itemize}

Output file name for plot is Gene1-Chromosome1\_Position1\_Gene2-Chromosome2\_Position2\_EventType.jpg
All the Outputs are written into a folder called \sphinxstylestrong{iAnnotateSVplots} in the given output directory

Please look at examples of input  and output files in /data/test directory where:
/data/test/testData.txt is the input file
/data/test/testResult.txt is the output file

The refFileVersion are automaticslly chosen from /data/references. \sphinxstylestrong{But caution this is only tested on hg19}. All these files are created using UCSC table browser.

The example for canonical transcripts can be also found in /data/canonicalInfo. In general the file is tab-delimited containing:

Gene  Transcripts

as the headers where:
\begin{itemize}
\item {} 
\sphinxstylestrong{Gene} : Gene symobol should match the gene name from  /data/references file.

\item {} 
\sphinxstylestrong{Transcripts} : Transcripts is a particular transcript that you are interested in using instead of auto-selection.

\end{itemize}

The file for hg19 uniprot is created using UCSC table browser (Uniprot spAnnot track). The file for hg19 is in /data/UcscUniprotdomainInfo


\subsection{Module \sphinxstyleliteralintitle{iAnnotateSV} contents}
\label{\detokenize{iAnnotateSV:module-iannotatesv-contents}}

\subsection{Submodules}
\label{\detokenize{iAnnotateSV:submodules}}

\subsection{\sphinxstyleliteralintitle{AnnotateEachBreakpoint} module}
\label{\detokenize{iAnnotateSV:annotateeachbreakpoint-module}}\begin{itemize}
\item {} \begin{description}
\item[{This module will annotate each breakpoint taking in:}] \leavevmode\begin{itemize}
\item {} 
\sphinxstylestrong{chr} : chromosome for the event,

\item {} 
\sphinxstylestrong{pos} : position in the chromosome for the event,

\item {} 
\sphinxstylestrong{str} : direction of the reads for the event{[}either 0 or 1{]},

\item {} 
\sphinxstylestrong{referenceDataframe} : a pandas data-frame that will store reference information

\end{itemize}
\begin{quote}\begin{description}
\item[{Example}] \leavevmode
\sphinxcode{AnnotateEachBreakpoint(chr1,pos1,str1,refDF)}

\end{description}\end{quote}

\end{description}

\end{itemize}


\subsection{\sphinxstyleliteralintitle{FindATranscript} module}
\label{\detokenize{iAnnotateSV:findatranscript-module}}\begin{itemize}
\item {} \begin{description}
\item[{This module will automatically find the highest preference transcript based on input:}] \leavevmode\begin{itemize}
\item {} \begin{description}
\item[{\sphinxstylestrong{queryDF}}] \leavevmode{[}Its a dataframe with{]}\begin{itemize}
\item {} 
\sphinxstylestrong{c} = zone: 1=exon, 2=intron, 3=3’-UTR, 4=5’-UTR, 5=promoter

\item {} 
\sphinxstylestrong{d,e} = for exons: which one, and how far

\item {} 
\sphinxstylestrong{d1,d2,e1,e2} = for introns: between which exons and how far?

\item {} 
\sphinxstylestrong{f} = for introns: how many bases in the partially completed codon?,

\end{itemize}

\end{description}

\item {} 
\sphinxstylestrong{referenceDataframe} : a pandas data-frame that will store reference information

\end{itemize}
\begin{quote}\begin{description}
\item[{Example}] \leavevmode
\sphinxcode{FindATranscript(queryDF,refDF)}

\end{description}\end{quote}

\end{description}

\end{itemize}


\subsection{\sphinxstyleliteralintitle{FindAllTranscripts} module}
\label{\detokenize{iAnnotateSV:findalltranscripts-module}}\begin{itemize}
\item {} \begin{description}
\item[{This module will find all transcripts based on input:}] \leavevmode\begin{itemize}
\item {} \begin{description}
\item[{\sphinxstylestrong{queryDF}}] \leavevmode{[}Its a dataframe with{]}\begin{itemize}
\item {} 
\sphinxstylestrong{c} = zone: 1=exon, 2=intron, 3=3’-UTR, 4=5’-UTR, 5=promoter

\item {} 
\sphinxstylestrong{d,e} = for exons: which one, and how far

\item {} 
\sphinxstylestrong{d1,d2,e1,e2} = for introns: between which exons and how far?

\item {} 
\sphinxstylestrong{f} = for introns: how many bases in the partially completed codon?,

\end{itemize}

\end{description}

\item {} 
\sphinxstylestrong{referenceDataframe} : a pandas data-frame that will store reference information

\end{itemize}
\begin{quote}\begin{description}
\item[{Example}] \leavevmode
\sphinxcode{FindAllTranscripts(queryDF,refDF)}

\end{description}\end{quote}

\end{description}

\end{itemize}


\subsection{\sphinxstyleliteralintitle{FindCanonicalTranscript} module}
\label{\detokenize{iAnnotateSV:findcanonicaltranscript-module}}\begin{itemize}
\item {} \begin{description}
\item[{This module will Finad a canonical transcript based on the input for main function and output of FindAllTranscripts:}] \leavevmode\begin{itemize}
\item {} 
\sphinxstylestrong{geneList} : List of genes {[}this will normally be list with same names{]} for the Structural Variant in question,

\item {} 
\sphinxstylestrong{transcriptList} : List of transcripts for the Structural Variant in question,

\item {} 
\sphinxstylestrong{siteList} : direction of the site for the event{[}either 0 or 1{]},

\item {} 
\sphinxstylestrong{zoneList} : different zones in which the event can occur {[}zone: 1=exon, 2=intron, 3=3’-UTR, 4=5’-UTR, 5=promoter{]}

\item {} 
\sphinxstylestrong{strandList} : direction of the read for the event{[}either 0 or 1{]},

\item {} 
\sphinxstylestrong{intronnumList} : Which intron the event occurs if the event is in intron for each transcript,

\item {} 
\sphinxstylestrong{intronframeList} : What is the frame of the intron where the event is occuring for each transcript.

\item {} 
\sphinxstylestrong{ctDict} : a dictionary containing the canonical transcript information for each gene. {[}Gene=\textgreater{}Transcript{]}

\end{itemize}
\begin{quote}\begin{description}
\item[{Example}] \leavevmode
\sphinxcode{FindCT(geneList,transcriptList,siteList,zoneList,strandList,intronnumList,intronframeList,ctDict)}

\end{description}\end{quote}

\end{description}

\end{itemize}


\subsection{\sphinxstyleliteralintitle{PredictFunction} module}
\label{\detokenize{iAnnotateSV:predictfunction-module}}\begin{itemize}
\item {} 
This module will predict the function of each annotated breakpoint

\item {} \begin{description}
\item[{It takes two pandas series which has following information:}] \leavevmode\begin{itemize}
\item {} 
\sphinxstylestrong{gene} : Gene for the event,

\item {} 
\sphinxstylestrong{transcript} : Transcript used for the event,

\item {} 
\sphinxstylestrong{site} : Explanation for site where the event occurs,

\item {} 
\sphinxstylestrong{zone} : Where does the event occur {[} 1=exon, 2=intron, 3=3’-UTR, 4=5’-UTR, 5=promoter {]},

\item {} 
\sphinxstylestrong{strand} : Direction of the transcript,

\item {} 
\sphinxstylestrong{str} : Direction of the read,

\item {} 
\sphinxstylestrong{intronnum} : Which intron the event occurs if the event is in intron,

\item {} 
\sphinxstylestrong{intronframe} : What is the frame of the intron where the event is occuring.

\end{itemize}
\begin{quote}\begin{description}
\item[{Example}] \leavevmode
\sphinxcode{ann1S = pandas.Series({[}gene1,transcript1,site1,zone1,strand1,str1,intronnum1,intronframe1{]},index={[}'gene1', 'transcript1', 'site1', 'zone1', 'txstrand1', 'readstrand1', 'intronnum1','intronframe1'{]})}

\sphinxcode{ann2S = pandas.Series({[}gene2,transcript2,site2,zone2,strand2,str2,intronnum2,intronframe2{]},index={[}'gene2', 'transcript2', 'site2', 'zone2', 'txstrand2', 'readstrand2', 'intronnum2','intronframe2'{]})}

So \sphinxstylestrong{ann1S} \& \sphinxstylestrong{ann2S} are series that will go to PredictFuntionForSV()

\sphinxcode{PredictFunctionForSV(ann1S,ann2S)}

\end{description}\end{quote}

\end{description}

\end{itemize}


\subsection{\sphinxstyleliteralintitle{AddExternalAnnotations} module}
\label{\detokenize{iAnnotateSV:addexternalannotations-module}}\begin{itemize}
\item {} \begin{description}
\item[{This module will annotate each breakpoint for Repeat Region, Database of Genomic Variants and Cosmic Census taking in:}] \leavevmode\begin{itemize}
\item {} 
\sphinxstylestrong{repeat region file} : Repeat Track from UCSC in tab-delimited format (see: \sphinxcode{/data/repeat\_region/hg19\_repeatRegion.tsv}),

\item {} 
\sphinxstylestrong{data base of genimic variant file} : DGv in tab-delimited format (see: \sphinxcode{/data/database\_of\_genomic\_variants/hg19\_DGv\_Annotation.tsv}),

\item {} 
\sphinxstylestrong{cosmic census file} : cosmic census file from sanger (see: \sphinxcode{/data/cosmic/cancer\_gene\_census.tsv}),

\item {} 
\sphinxstylestrong{cosmic fusion counts file} : cosmic fusion counts file from from cosmic fusion export (see: \sphinxcode{/data/cosmic/cosmic\_fusion\_counts.tsv}),

\item {} 
\sphinxstylestrong{structural variants file} : File containing the breakpoint information,

\item {} 
\sphinxstylestrong{output prefix} : Output Prefix for the output files (.xlsx,.json,.txt),

\item {} 
\sphinxstylestrong{output directory} : Directory where the output needs to be written

\end{itemize}
\begin{quote}\begin{description}
\item[{Example}] \leavevmode
\sphinxcode{makeCommandLineForAEA = "-r " + repeatregionFilePath + " -d " + dgvFilePath + " -c " + ccFilePath + " -cct " + cctFilePath + " -s " + svFilePath + " -ofp AnnotatedSV" + " -o " + outputDir}

\sphinxcode{AddExternalAnnotations.main(makeCommandLineForAEA)}

\end{description}\end{quote}

\end{description}

\end{itemize}


\subsection{\sphinxstyleliteralintitle{AnnotateForRepeatRegion} module}
\label{\detokenize{iAnnotateSV:annotateforrepeatregion-module}}\begin{itemize}
\item {} \begin{description}
\item[{This module has two submodules will read and annotate each breakpoint for Repeat Region}] \leavevmode\begin{enumerate}
\item {} 
\sphinxstylestrong{ReadRepeatFile}
\begin{itemize}
\item {} 
This will read a tab-delimited file into a panadas dataframe

\end{itemize}

\item {} 
\sphinxstylestrong{AnnotateRepeatRegion}
\begin{quote}
\begin{itemize}
\item {} 
This is will annotate the breakpoints for repeat region.

\end{itemize}
\begin{quote}\begin{description}
\item[{Example}] \leavevmode
\sphinxcode{AnnotateRepeatRegion(verbose, count, svObject, repeatregionDict)}

\end{description}\end{quote}
\end{quote}

\end{enumerate}

\end{description}

\end{itemize}


\subsection{\sphinxstyleliteralintitle{AnnotateForDGv} module}
\label{\detokenize{iAnnotateSV:annotatefordgv-module}}\begin{itemize}
\item {} \begin{description}
\item[{This module has two submodules will read and annotate each breakpoint for Database of Genomic Variants}] \leavevmode\begin{enumerate}
\item {} 
\sphinxstylestrong{ReadDGv}
\begin{itemize}
\item {} 
This will read a tab-delimited file into a panadas dataframe

\end{itemize}

\item {} 
\sphinxstylestrong{AnnotateDGv}
\begin{quote}
\begin{itemize}
\item {} 
This is will annotate the breakpoints for Database of Genomic Variants.

\end{itemize}
\begin{quote}\begin{description}
\item[{Example}] \leavevmode
\sphinxcode{AnnotateDGv(verbose, count, svObject, dgvDict)}

\end{description}\end{quote}
\end{quote}

\end{enumerate}

\end{description}

\end{itemize}


\subsection{\sphinxstyleliteralintitle{AnnotateForCosmic} module}
\label{\detokenize{iAnnotateSV:annotateforcosmic-module}}\begin{itemize}
\item {} 
This module will annotate each breakpoint for Cosmic Census
\begin{quote}
\begin{quote}\begin{description}
\item[{Example}] \leavevmode
\sphinxcode{AnnotateFromCosmicCensusFile(comic\_census\_filename, verbose, count, svObject)}

\item[{Example}] \leavevmode
\sphinxcode{AnnotateFromComicFusionCountsFile(comic\_fusion\_counts\_filename, verbose, count, svObject)}

\end{description}\end{quote}
\end{quote}

\end{itemize}


\subsection{\sphinxstyleliteralintitle{helper} module}
\label{\detokenize{iAnnotateSV:helper-module}}\begin{itemize}
\item {} \begin{description}
\item[{This module has multiple submodules}] \leavevmode\begin{enumerate}
\item {} 
\sphinxstylestrong{ReadFile()}
\begin{itemize}
\item {} 
This will read a tab-delimited file into a panadas dataframe

\end{itemize}

\item {} 
\sphinxstylestrong{ExtendPromoterRegion()}
\begin{itemize}
\item {} 
This will extend the promoter region to a given length

\end{itemize}

\item {} 
\sphinxstylestrong{bp2str()}
\begin{itemize}
\item {} 
This will convert base pair information to string information

\end{itemize}

\end{enumerate}

\end{description}

\end{itemize}


\subsection{\sphinxstyleliteralintitle{VisualizeSV} module}
\label{\detokenize{iAnnotateSV:visualizesv-module}}\begin{itemize}
\item {} \begin{description}
\item[{This module will annotate each breakpoint taking in:}] \leavevmode\begin{itemize}
\item {} 
\sphinxstylestrong{svDataFrame} : Annotated structurla varaints dataframe obtained from PredictFuntion Module,

\item {} 
\sphinxstylestrong{referenceDataFrame} : a pandas data-frame that will store reference information,

\item {} 
\sphinxstylestrong{uniprotDataFrame} : making a dataframe from the uniprot data.

\item {} 
\sphinxstylestrong{args} : This has all the arguments that are generated from iAnnotateSV module

\end{itemize}
\begin{quote}\begin{description}
\item[{Example}] \leavevmode
\sphinxcode{VisualizeSV(svDataFrame,referenceDataFrame,uniprotDataFrame,args)}

\item[{Example Plot}] \leavevmode
\noindent{\hspace*{\fill}\scalebox{1.000000}{\sphinxincludegraphics[width=300\sphinxpxdimen,height=300\sphinxpxdimen]{{EWSR1-chr22_29688289_ERG-chr21_39775034_Translocation}.jpg}}\hspace*{\fill}}

\end{description}\end{quote}

\end{description}

\end{itemize}


\subsection{\sphinxstyleliteralintitle{iAnnotateSV} module}
\label{\detokenize{iAnnotateSV:iannotatesv-module}}\begin{itemize}
\item {} 
This module is the driver module, it takes user information and runs all other module to produce proper structural variant annotation

\end{itemize}

\fvset{hllines={, ,}}%
\begin{sphinxVerbatim}[commandchars=\\\{\}]
usage: iAnnotateSV.py \PYG{o}{[}options\PYG{o}{]}**

Annotate SV based on a specific human reference**

optional arguments:**
\PYGZhy{}h, \PYGZhy{}\PYGZhy{}help            show this \PYG{n+nb}{help} message and \PYG{n+nb}{exit}
\PYGZhy{}v, \PYGZhy{}\PYGZhy{}verbose         make lots of noise \PYG{o}{[}default\PYG{o}{]}
\PYGZhy{}r hg19, \PYGZhy{}\PYGZhy{}refFileVersion hg19
                        Which human reference file to be used, hg18,hg19 or
                        hg38
\PYGZhy{}ofp test, \PYGZhy{}\PYGZhy{}outputFilePrefix \PYG{n+nb}{test}
                        Prefix \PYG{k}{for} the output file
\PYGZhy{}o /somedir, \PYGZhy{}\PYGZhy{}outputDir /somedir
                        Full Path to the output dir
\PYGZhy{}i svfile.txt, \PYGZhy{}\PYGZhy{}svFile svfile.txt
                        Location of the structural variants file to annotate
\PYGZhy{}d \PYG{l+m}{3000}, \PYGZhy{}\PYGZhy{}distance \PYG{l+m}{3000}
                        Distance used to extend the promoter region
\PYGZhy{}a, \PYGZhy{}\PYGZhy{}autoSelect      Auto Select which transcript to be used\PYG{o}{[}default\PYG{o}{]}
\PYGZhy{}c canonicalExons.txt, \PYGZhy{}\PYGZhy{}canonicalTranscripts canonicalExons.txt
                        Location of canonical transcript list \PYG{k}{for} each gene.
                        Use only \PYG{k}{if} you want the output \PYG{k}{for} specific
                        transcripts \PYG{k}{for} each gene.
\PYGZhy{}p, \PYGZhy{}\PYGZhy{}plotSV          Plot the structural variant in question\PYG{o}{[}default\PYG{o}{]}
\PYGZhy{}u uniprot.txt, \PYGZhy{}\PYGZhy{}uniprotFile uniprot.txt
                        Location of UniProt list contain information \PYG{k}{for}
                        protein domains. Use only \PYG{k}{if} you want to plot the
                        structural variant
\PYGZhy{}rr RepeatRegionFile.tsv, \PYGZhy{}\PYGZhy{}repeatFile RepeatRegionFile.tsv
                        Location of the Repeat Region Bed File
\PYGZhy{}dgv DGvFile.tsv, \PYGZhy{}\PYGZhy{}dgvFile DGvFile.tsv
                        Location of the Database of Genomic Variants Bed File
\PYGZhy{}cc CosmicConsensus.tsv, \PYGZhy{}\PYGZhy{}cosmicConsensusFile CosmicConsensus.tsv
                        Location of the Cosmic Consensus TSV file
\PYGZhy{}cct CosmicFusionCounts.tsv, \PYGZhy{}\PYGZhy{}cosmicCountsFile CosmicConsensus.tsv
                        Location of the Cosmic Fusion Counts TSV file
\end{sphinxVerbatim}


\chapter{Citation}
\label{\detokenize{index:citation}}
We are in the process of publishing a manuscript describing iAnnotateSV as part of the Structural Variant Detection framework.
If you use this software in a publication, for now, please cite our website \sphinxhref{http://github.com/rhshah/iAnnotateSV}{iAnnotateSV}.


\chapter{Acknowledgements}
\label{\detokenize{index:acknowledgements}}
I would like to thanks Mike Lawrence from Braod Institue for sharing his code and Michael Berger for his inshgts into the dRanger\_Annoate tool.


\chapter{Indices and tables}
\label{\detokenize{index:indices-and-tables}}\begin{itemize}
\item {} 
\DUrole{xref,std,std-ref}{genindex}

\item {} 
\DUrole{xref,std,std-ref}{modindex}

\item {} 
\DUrole{xref,std,std-ref}{search}

\end{itemize}



\renewcommand{\indexname}{Index}
\printindex
\end{document}